% !TEX root = epifanov_solid_state_physics.tex
%!TEX TS-program = pdflatex
%!TEX encoding = UTF-8 Unicode


%\chapter*{Bibliography}
%\addcontentsline{toc}{chapter}{Bibliography}
%\chaptermark{Bibliography}

\vspace*{-12pt}

{\Large{\sffamily\bfseries OTHER MIR TITLES}}

\bigskip
\bigskip
\bigskip

\noindent
\textbf{Fundamentals of the Theory of Electricity}

\noindent
I. TAMM, Mem. USSR Acad. Sci.

\noindent
The present book is mainly intended for students of physical faculties of universities who have mastered differential and integral calculus and vector algebra; the fundamentals of vector analysis are set out in the text as needed. The main object of this course is the determination of the physical meaning and content of the main laws of the theory of electricity. Although not treating the technical applications of the theory, the author prepares the reader as far as possible to a direct transition to studying the applied theory of electricity. The book contains a number of problems forming an organic part of the text. The solutions of the majority of them are needed for an understanding of the subject matter.

\bigskip
\bigskip
\bigskip

\noindent
\textbf{Special Theory of Relativity}

\noindent
V. UGAROV, Cand. Sc.

\noindent
This is the English translation of the second revised and enlarged edition of the book which was first published in 1969 under the same title. It is written as a textbook to be used both by university students and teachers as well as high school teachers. Although the general layout of the book has not been changed, the principles of the theory are given in more details now and the greater attention is paid to the four-dimensional treatment. Different ways of presenting the special theory of relativity are described. Sections devoted to the methodology and history of the special theory of relativity are added. The chapter on electrodynamics is enlarged. Finally the paper written by Academician V. L. Ginzburg and entitled ``Who Created the Special Theory of Relativity and How It Was Done'' is included.


\clearpage

\vspace*{5cm}

\parindent=10pt

\centering{To the Reader}

\vspace*{0.5cm}

\begin{minipage}[c]{0.65\textwidth}

    $\quad$Mir Publishers welcome your comments on the content, translation and design of this book.

    $\quad$We would also be pleased to receive any proposals you care to make about our future publications.

    $\quad$Our address is:

    USSR, 129820, Moscow 1-110, GSP,

    Pervy Rizhsky Pereulok, 2

    Mir Publishers

\end{minipage}

\bigskip
\bigskip

\textit{Printed in the Union of Soviet Socialist Republics}

\vfill{ }

%\pagestyle{mystyle}
